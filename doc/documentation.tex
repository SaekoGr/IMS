% Author: Dominik Harmim <xharmi00@stud.fit.vutbr.cz>
% Author: Vojtěch Hertl <xhertl04@stud.fit.vutbr.cz>


\documentclass[a4paper, 11pt]{article}


\usepackage[czech]{babel}
\usepackage[utf8]{inputenc}
\usepackage[left=2cm, top=3cm, text={17cm, 24cm}]{geometry}
\usepackage{times}
\usepackage{graphicx}

\begin{document}
\begin{titlepage}
\begin{center}

\vspace{\stretch{0.382}}

\Huge{Simulačná štúdia} \\
\LARGE{\textbf{Uhlíková stopa v doprave}} \\

\vspace{\stretch{0.618}}
\end{center}

\begin{minipage}{0.5 \textwidth}
\Large
\today
\end{minipage}
\hfill
\begin{minipage}[r]{0.5 \textwidth}
\Large
\begin{tabular}{ll}
\textbf{Sabína Gregušová} & \textbf{(xgregu02)} \\
Filip Weigel & (xweige01)
\end{tabular}
\end{minipage}
\end{titlepage}

\clearpage
\tableofcontents


\clearpage
\pagenumbering{arabic}
\setcounter{page}{1}

\section{Úvod}
Cílem této Simulační studie je sestavit model středně velkého evropske města a simulovat pohyb osob dopravními prostředky, které produkuj emise CO2. Investice do infrastruktury a nákupy nových vozidel představuj nemalé Peníze pohybujících se v řádek desítek, stovek milionu, případně i miliard korun. Pomocí simulace jednoho, či více pracovních dní budeme moci pozorovat, jak mohou rozhodnutí jednotlivců přispět k tvorbě uhlíkové stopy a budeme se snažit najít optimální řešení, které by mohlo pomoci k jejich snížení. Vyplatí se investovat do nových autobusů a vozidel, které mají nižší emise? Jak se změní emise CO2, Pokud provozovatel MHD snižuje cenu jízdného a bude hromadnou dopravu využívat více obyvatel? Tyto a mnoho dalších otázek můžeme díky simulaci zodpovědět. 

\section{Rozbor tématu a fakta}
Uhlíková stopa se stala v poslední době fenoménu, který pohltil téměř celý svět. Téměř v každém koutu světa se konají summity na téma znečištění ovzduší automobily. V Evropě tvoří 30 \% z celkové produkce CO2 doprava a transport zboží na základě průzkumu Evropského parlamentu z roku 2017 \cite {co2_eu}. Evropská Unie se podílí produkci oxidu uhličitého 13 \% z celého světa. Paradoxně nejvýrazněji bojovníci je Greta Thunberg pocházející ze Švédska, která šokovala svět při svém projevu. Apelovala na občany všech zemí, že by Melia přestat jezdit automobily, neboť osobní vozidla způsobují asi 60.7 \% CO2 emisí z celkového množství emisí, které produkuje doprava \cite {co2_eu}, což je asi 12 \% z celkového množství emisí, které produkuje Evropa \cite {co2_eu_law}.

V rámci fakt je třeba si taktéž přiblížit princip spalovacího motoru. Spalovací motor je druh stroje, který přeměňuje chemickou energii obsaženou v palivě především na teplo a mechanickou energii. Teplo tvoří asi 75 \% přeměněné energie a je v automobilovém průmyslu využito jen pro vytápění prostoru pro cestující. Je to tedy nechtěný produkt. Zbylých 25 \% je přeměněno na mechanickou práci. Hlavní částí motoru je válec, píst, ventily umístěné v hlavě motoru a kliková hřídel. Všechny ze zmíněných součástí tvoří tzv. spalovací prostor.
Píst se pohybuje ve válci nahoru a dolů v posuvném pohybu. V případě, že je píst ve válci úplně nahoře, tak tuto událost nazýváme horní úvrať, pokud je píst úplně dole tak dolní úvrať. Při posuvném pohybu pístu se zároveň otáčí kliková hřídel, která převádí posuvný pohyb na otáčivý. 

Budeme uvažovat obyčejný atmosférický motor, který nemá žádné pomocné součásti typu turbodmychadlo a kompresor. 
Fáze:
\begin{enumerate}
    \item Píst nachází v horní úvrati a posunuje se směrem dolů a nasává vzduch otevřeným sacím ventilem.
    \item Píst pohybuje směrem nahoru, stlačuje nasátý vzduch v spalovacím prostoru a způsobuje jeho zahřátí.
    \item Těsně před horní úvratí se pomocí vstřikovače stříkne do spalovacího prostoru přesně odměřená dávka paliva, která se vznítí a tlačí píst dolů. Zde dochází k velmi patrné ztrátě účinnosti, jelikož palivo vybuchuje ještě ve fázi, kdy píst jde nahoru.
    \item Píst je setrvačností tlačen nahoru a probíhá výfuk. Výfukový ventil je otevřen a ven ze spalovacího prostoru jsou tlačeny spaliny.
\end{enumerate}

Spaliny jsou z valné většiny tvořeny oxidem uhličitým a různými prvky/sloučeninami, jež jsou produkty hoření paliva.

Evropská unie se snaží redukovat produkci CO2 emisí nových modelů osobních automobilů pod hranici 130 g/Km; v některých zemích se již podařilo snížit tento průměr na 120.4 g/Km. Emise CO2 má tedy při různých modelech osobních automobilů různou odchylku. Musíme brát v úvahu, že tento nový zákon se týká nových osobních automobilů, zatímco lidé stále využívají osobní automobily, které byly vyrobeny více než 10 let dozadu. Můžeme se proto domnívat, že skutečná průměrná produkce CO2 se pohybuje okolo 135 g/Km \cite{ultimatespecs.com}.


506/5000
Městská hromadná doprava také produkuje CO2 emise, avšak dokáže přepravit několikanásobně více osob. V této studii se zabýváme dopravními prostředky pro přepravu osob ve městě, které produkují emise CO2 pomocí spalovacího motoru. Mnohá města disponují rozsáhlou sítí městské hromadné dopravy, která se skládá z tramvají, tramvají a autobusů. Autobusy disponují spalovacím motorem, a na základě průzkumu vyprodukují asi 822 g/Km s mírnými odchylkami různých modelů \cite {bust_travel}.



\section{Koncepce modelu}

Cílem modelu, viz Petriho síť, je simulovat běžný pracovní den v středně velkém evropském městě, a jak rozhodnutí jednotlivců ovlivňují celkovou denní produkci CO2 emisí. Vstupním parametrem modelu je počet osob, které mají být přepraveny vrámci jednoho dne. Pro zjednodušení modelu předpokládáme, že vrámci tohoto jednoho dne musí být úspěšně přepraveny všechny osoby. V Evropské unii vlastní asi 60\% obyvatelstva alespoň 1 vozidlo, proto předpokládáme, že každá osoba má 60\% šanci, že vlastní alespoň jedno vozidlo \cite {rss}. 

Osoby, které auto nevlastní jdou hned na zastávku autobusu MHD. Předpokládáme, že osoby, které auto vlastní se rozhodnou použít MHD s pravděpodobností přibližně 25\% na základě průzkumu. Tento údaj vznikl zprůměrováním hodnot evropských zemí, kde respondenti řekli, že využijí MHD 1 a více krát za týden. Výsledný údaj byl 37.2\%, musíme však brát v úvahu, že tento údaj se vztahuje na všechny občany, takže odhadujeme, že asi 25\% lidí, co vlastní osobní automobil využívají MHD. Toto procento je použito pouze na verifikaci, protože jeho hodnota je vstupním parametrem modelu, protože chceme být schopni simulovat situaci, kdy se MHD stane velmi žádoucí (například rapidně snížení ceny) a toto procento se bude tehdy měnit.



\section{Architektúra simulačního modelu}
Zdroj uvádí, že v roce 2015 se za rok v celé ČR vyprodukovalo celkem 6 634 769 tun CO2, které byly vyprodukovány individuální osobní dopravou a městskou hromadnou dopravou. Předpokládejme tedy, že se vyprodukuje 18 177 tun CO2 denně. Pokud toto číslo vydělíme počtem obyvatel, dostaneme číslo které vypovídá o znečistění jednoho obyvatele na den. Z výpočtu vyplývá, že každý obyvatel ČR zanechává uhlíkovou stopu rovnu 173,1 gramu. 

Průměrný počet zastávek autobusu: 28 zastávek, průměrná vzdálenost mezi zastávkami 550 metrů – vypočítáno na základě 10 linek MHD v Brně. Doba potřebná pro ujetí jednoho km: 97 sekund – průměrná rychlost ve větším městě: 37 km/h. 

Keďže chceme simulovať pohyb ľudí čo najprirodzenejším spôsobom, lambda pre exponenciálnu pravdepodobnosť sa počíta dynamicky. Keďže je naším cieľom, aby sa prepravili všetci ľudia do konca simulácie, je potrebné obmedziť generovanie nových ľudí (procesov) a počítať s určitou rezervou. Experimentálne sme zistili, že ľudia sa musia prestať generovať približne 2 hodiny a 55 minút pred ukončením simulácie, aby záujemcovia o prepravu autobusom stihli nastúpiť ne nejaký autobus a nezostali čakať na zastávkach. Lambdou pre našu exponenciálnu funkciu je teda 75900 (počet sekúnd, vrámci ktorých sa musia vygenerovať všetci ľudia) delený celkovým počtom obyvateľov.

Šanca, že daný človek vlastní auto je taktiež kalkulovaná dynamicky, aby zodpovedala zadaným vstupom. Toto percento je kalkulované ako počet všetkých áut deleno počet všetkých ľudí, to jest koľko áut pripadá na jedného človeka. Ak z pravdepodobnosti výjde, že človek má auto, musia byť stále nejaké autá dostupné, inak auto nemá. Ak aj človek auto má, rozhodne sa podľa vstupného argumentu transportRatio (defaultná hodnota je 25\%), či použije mestskú hromadnú dopravu, alebo použije svoje auto. 



Časový údaj, kdy se má generovat autobus je dán číslem 79 200 vyděleným počtem autobusů, abychom docílili stavu, že se vygenerují vždy všechny autobusové okruhy s dostatečnou časovou rezervou. 

Okruh je tedy vygenerován a autobus stojí v depu. Předpokládáme, že autobus je třeba nastartovat, tlakovat vzduchové okruhy a dojet na první zastávku. Pro zmíněný účel poslouží normální rozdělení se střední hodnotou 500 sekund. Autobus dojel na první zastávku a cestující nastupují. průměrný čas nastoupení jednoho cestujícího je dán normálním rozdělením se středem 2,5s. V modelu nastupuje jede pasažér po druhém. Model nereflektuje situaci, že reálný autobus má více dveří, proto je zmíněná hodnota nižší. 
Po nastoupení a vystoupení všech cestujících se autobus rozjede na další zastávku. Doba jízdy je dána normálním rozdělením se středem 84 sekund. Autobus poté jezdí v cyklu mezi zastávkami. Na konečné zastávce jsou všichni cestující nuceni vystoupit a autobus jede zpět do depa. Dle ujeté vzdálenosti model počítá, kolik CO2 autobus/automobil vyprodukoval a číslo je přičteno k celkovému znečistění za den. Na konci simulace je vypsána statistika o celkovém znečistění.


\section{Verifikace a experimenty}
\subsection{Verifikace}
\section{Závěr}

\newpage
\bibliographystyle{czechiso}
\bibliography{documentation}

\end{document}