% Author: Dominik Harmim <xharmi00@stud.fit.vutbr.cz>
% Author: Vojtěch Hertl <xhertl04@stud.fit.vutbr.cz>


\documentclass[a4paper, 11pt]{article}


\usepackage[czech]{babel}
\usepackage[utf8]{inputenc}
\usepackage[left=2cm, top=3cm, text={17cm, 24cm}]{geometry}
\usepackage{times}
\usepackage{graphicx}

\begin{document}
\begin{titlepage}
\begin{center}

\vspace{\stretch{0.382}}

\Huge{Simulačná štúdia} \\
\LARGE{\textbf{Uhlíková stopa v doprave}} \\

\vspace{\stretch{0.618}}
\end{center}

\begin{minipage}{0.5 \textwidth}
\Large
\today
\end{minipage}
\hfill
\begin{minipage}[r]{0.5 \textwidth}
\Large
\begin{tabular}{ll}
\textbf{Sabína Gregušová} & \textbf{(xgregu02)} \\
Filip Weigel & (xxxxxxxx)
\end{tabular}
\end{minipage}
\end{titlepage}

\clearpage
\tableofcontents


\clearpage
\pagenumbering{arabic}
\setcounter{page}{1}

\section{Úvod}
Cílem této Simulační studie je sestavit model středně velkého evropske města a simulovat pohyb osob dopravními prostředky, které produkuj emise CO2. Investice do infrastruktury a nákupy nových vozidel představuj nemalé Peníze pohybujících se v řádek desítek, stovek milionu, případně i miliard korun. Pomocí simulace jednoho, či více pracovních dní budeme moci pozorovat, jak mohou rozhodnutí jednotlivců přispět k tvorbě uhlíkové stopy a bude se snažit najít optimální řešení, které by mohlo pomoci k jejich snížení. Vyplatí se investovat do nových autobusů a vozidel, které mají nižší emise? Jak se změní emise CO2, POkud provozovatel MHD snižuje cenu jízdného a bude hromadnou dopravu využívat více obyvatel? Tyto a mnoho dalších otázek Můžeme díky simulaci zodpovědět.

\section{Rozbor tématu a fakta}
Uhlíková stopa se stala v poslední době fenoménem, který pohltil téměř celý svět. Téměř v každém koutu světa se konají summity na téma znečisťování ovzduší automobily. V Evropě tvoří 30 \% z celkové produkce CO2 doprava a transport zboží na základe prieskumu Európskeho parlamentu z roku 2017[EPčlanok]. Evropská Unie se podílí produkci oxidu uhličitého 13 \% z celého světa. Paradoxně nejvýraznější bojovnicí je Greta Thunberg pocházející ze Švédska, která šokovala svět při svém projevu. Apelovala na občany všech zemí, že by měli přestat jezdit automobily.

\section{Koncepce modelu}
\section{Architektúra simulačního modelu}
\section{Experimenty}
\section{Závěr}

\newpage
%\bibliographystyle{czechiso}
%\bibliography{documentation}

\end{document}